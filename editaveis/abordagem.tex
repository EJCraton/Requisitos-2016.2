\chapter{Abordagem da Engenharia de Requisitos}

\section{SAFe}
Scaled Agile Framework, ou SAFe, é utilizado para dimensionar processos ágeis para grandes projetos. Uma das filosofias do processo ágil é uma equipe pequena auto gerenciavel, porém, por meio do SAFe, é possível garantir esse nível de auto gerenciamento em uma grande quantidade de equipes trabalhando em um mesmo projeto.

Como o SAFe é baseado em processos ágeis, o framework também compartilha os princípios ágeis. Como por exemplo, o uso do kanban para os diferentes níveis, e até mesmo a utilização do scrum no nível mais baixo.

Além de compartilhar essas características com os processos ágeis, o SAFe possui 9 príncipos, são eles:
\begin{enumerate}
\item Ter uma visão econômica
\item Aplicar um pensamento sistêmico
\item Assumir variabilidade; preservar opções
\item Construir incrementalmente de forma rapida, com ciclos integrados de aprendizagem
\item Se baseie em marcos com o objetivo de avaliação de sistemas funcionais
\item Visualize e limite trabalhos em andamento, reduzir a quantidade de trabalhos e gerenciar grandes filas de espera.
\item Utilizar uma cadencia e sincronizar planejamento entre domínios
\item Habilitar a motivação intrínseca de conhecimento dos trabalhadores
\item Descentralizar a tomada de decisão
\end{enumerate}


O SAFe é divido em 3 níveis: \textbf{portfólio, programa e time.}


\subsection{Portfólio}
De acordo com o site do framework, o nível de portfólio é voltado para as pessoas e processos que são necessárias para a construção de sistemas e soluções que a empresa necessita para atingir uma meta prevista. 

\subsection{Programa}
O nível de programa é onde os times de desenvolvimento e outros recursos são aplicados em alguma solução em desenvolvimento corrente.

\subsection{Time}
Por fim, o nível de time mesmo que definido como níveis diferentes, é uma parte do nível de programa. Cada time é responsável por definir, construir e testar as histórias de seu backlog de time dentro de sprints, e cada um é organizado de acordo com as capacidades e talentos de cada indivíduo.

\subsection{Big picture}
\begin{figure}[!htpb]
	\centering
	\includegraphics[scale=0.7]{figuras/abordagem/SAFe_Big_Picture_4}
	\caption{Big picture do SAFe}
\end{figure}

\section{Tradicional}

\section{Híbrido}

\section{Justificativa}
