\chapter{Processo de engenharia de requisitos}

\section{Maturidade}

\subsection{CMMI (Capability Maturity Model $-$ Integration)}

  O CMMI é um modelo de maturidade internacional, geralmente utilizado por empresas
  globalizadas que necessitam de certificação para serem reconhecidos internacionalmente.
  Surgiu como integração e evolução dos modelos SW-CMM (Capability Maturity Model for Software),
  SECM EIA 731 (System Engineering Capability Model) e  e IPD-CMM
  (Integrated Product Development CMM).\cite{mct2006}.

  O CMMI possui 5 níveis de maturidade, enumerdos do 1 ao 5 e cada nível corresponde
  à capacidade da organização de desenvolvimento do produto, sendo o nível 1 o mais
  baixo nível de maturidade e o nível 5 o mais alto e contínuo.

  \begin{figure}[!ht]
    \centering
    \includegraphics[width=15cm, keepaspectratio=true]{figuras/maturidade/niveis-cmmi.eps}
    \caption{Os cinco níveis de maturidade do modelo CMMI}
  \end{figure}

\subsubsection{Vantagens}

  Como já dito na descrição na Subseção 1.1, o CMMI é um modelo de melhoria
  de processos reconhecido internacionalmente, fazendo com que a organização
  certificada por ele tenha a vantagem de ser reconhecida em qualquer lugar
  do mundo por causa da aplicação do CMMI

  Com o CMMI, a organização vai se otimizando cada vez mais e o controle do
  processo fica cada vez mais definido, sendo assim, a organização continuamente
  vai identificando o que realmente tem valor de acordo com sua maturidade.

\subsubsection{Desvantagens}

  O CMMI tem um alto custo, por isso geralmente quem tem a certificação são
  grandes organizações globalizadas que possuem recursos para sustentar a
  avaliação e se torna vantajoso para seu reconhecimento internacional. O
  tempo para amadurecimento do processo pode levar de 4 a 8 anos, sendo assim
  definido um modelo moroso e caro. Segundo Franciscani, algumas organizações
  tratam o CMMi como um processo e não como modelo, fazendo com que eles tenham
  a percepção de que nem tudo que está no CMMI seja mesmo necesário.\cite{francis2012}.

\subsection{MPS.BR}

  O modelo MPS.BR foi desenvolvido pela Softex com o objetivo de atingir
  certificações de pequenas e médias empresas, bem como possibilitar que
  as empresas possam ter acesso mais facilitado para a certificação.
  O modelo MPS.BR se adequa ao mercado brasileiro de software e deriva do CMMI
  Enquanto o CMMI possui cinco níveis de maturidade enumeradas do 1 ao 5, o
  MPS.BR possui sete níveis classificados, de forma piramidal,  pelas letras
  do A ao G, sendo o nível A o mais alto e contínuo nível de maturidade e o G
  o mais baixo. A Figura 2 contém a representação dos níveis de maturidade do
  MPS.BR

  \begin{figure}[!ht]
    \centering
    \includegraphics[width=15cm, keepaspectratio=true]{figuras/maturidade/niveis-mpsbr.eps}
    \caption{Os sete níveis de maturidade do modelo MPS.Br}
  \end{figure}

\subsubsection{Vantagens}

  Além do MPS.BR ser mais acessível e estar mais adequado ao contexto de
  organizações brasileiras, segundo Franciscani, existem outras vantagens
  como:

  \begin{itemize}
    \item{Compatibilidade com CMMI, podendo ser aproveitado para uma futura
          certificação nesse modelo.}
    \item{Dois números de nível a mais do que do CMMI, o que pode facilitar a
          implantação em pequenas e médias organizações.}
    \item{Obrigatoriedade do certificado em licitações.}
    \item{Integração Universidade-Empresa.}
  \end{itemize}

\subsubsection{Desvantagens}

  O MPS.BR não possibilita as empresas serem competitivas internacionalmente,
  por ser um modelo que se adequa apenas para certificação nacional. Isso pode
  ser uma desvantagem muito grande para organizações que pretendem
  globalizar-se.\cite{francis2012}.

\subsection{Seleção do Modelo}

  A Cráton é uma empresa júnior com apenas 15 membros formada por estudantes no qual
  não visam lucro para si mesmos, pois todo o dinheiro obtido deve ser investido
  na própria empresa como consta na lei número 13.267 de 2016 que regulamenta as empresas
  juniores e as define com fim educacional e não lucrativo. Neste contexto, o CMMI seria
  impraticável devido ao alto custo e a limitação de crescimento de uma empresa júnior que é
  completamente dependente das universidades e não seria globalizada, não precisando do CMMI

  O grupo possui quatro integrantes, o que se encaixa no contexto do MPS.BR que se direciona
  principalmente para organizações menores, fazendo-se ser mais acessível para poder desenvolver
  o modelo dentro do contexto do cliente.

  O cliente, sendo uma empresa júnior integrada com a Universidade de Brasília, poderia se
  beneficiar dos processos do MPS.BR, já que é um modelo que possui integração universidade-empresa.
  Dessa forma, o modelo seria aplicado diretamente no meio acadêmico.

  Por esses motivos apontados, o modelo MPS.BR foi selecionado para a engenharia de requisitos da
  solução do problema da Cráton.

\subsection{Processos Selecionados}

  Para o contexto da disciplina, serão implementados dois processos que tratam de
  requisitos, sendo o processo de Gerência de Requisitos que se encontra no nível
  G do MPS.BR e o Desenvolvimento de Requisitos que é um processo do nível D. Nos
  subtópicos seguintes, 1.4.1 e 1.4.2, estará o propósito e os resultados esperados
  de cada processo referenciado diretamente do Guia Geral do MPS.BR.\cite{softexmps}.

\subsubsection{Gerência de Requisitos $-$ GRE}

  \begin{description}
    \item[Propósito] \
      O propósito do processo Gerência de Requisitos é gerenciar os requisitos do
      produto e dos componentes do produto do projeto e identificar inconsistências
      entre os requisitos, os planos do projeto e os produtos de trabalho do projeto.
    \item [Resultados Esperados] \
      \begin{itemize}
        \item GRE 1. O entendimento dos requisitos é obtido junto aos fornecedores de requisitos;
        \item GRE 2. Os requisitos são avaliados com base em critérios objetivos e um comprometimento
              da equipe técnica com estes requisitos é obtido;
        \item GRE 3. A rastreabilidade bidirecional entre os requisitos e os produtos de trabalho
              é estabelecida e mantida;
        \item GRE 4. Revisões em planos e produtos de trabalho do projeto são realizadas;
              visando identificar e corrigir inconsistências em relação aos requisitos;
        \item GRE 5. Mudanças nos requisitos são gerenciadas ao longo do projeto.
      \end{itemize}
    \end{description}

\subsubsection{Desenvolvimento de Requisitos $-$ DRE}

  \begin{description}
    \item [Propósito] \
     O propósito do processo Desenvolvimento de Requisitos é definir os requisitos
     do cliente, do produto e dos componentes do produto.
    \item [Resultados Esperados]\
      \begin{itemize}
        \item DRE 1. As necessidades, expectativas e restrições do cliente, tanto do produto
              quanto de suas interfaces, são identificadas;
        \item DRE 2. Um conjunto definido de requisitos do cliente é especificado e priorizado
              a partir das necessidades, expectativas e restrições identificadas;
        \item DRE 3. Um conjunto de requisitos funcionais e não-funcionais, do produto e dos componentes
                    do produto que descrevem a solução do problema a ser resolvido, é definido e mantido
                    a partir dos requisitos do cliente;
        \item DRE 4. Os requisitos funcionais e não-funcionais de cada componente do produto são
              refinados, elaborados e alocados;
        \item DRE 5. Interfaces internas e externas do produto e de cada componente do produto são
              definidas;
        \item DRE 6. Conceitos operacionais e cenários são desenvolvidos;
        \item DRE 7. Os requisitos são analisados, usando critérios definidos, para balancear
              as necessidades dos interessados com as restrições existentes;
        \item DRE 8. Os requisitos são validados.
      \end{itemize}
  \end{description}

