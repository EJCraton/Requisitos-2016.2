\chapter{Elicitação de requisitos}

\section{Definição}
  Elicitação de requisitos é a parte inicial e talvez a mais essencial do início de um sistema de software, e é de suma importância que tanto analista como desenvolvedor tenha essa consciência. Basicamente, a etapa de elicitação de requisitos se resume no entendimento do problema, ou seja, deseja-se ao máximo conseguir extrair informações do cliente para o produto desejado.
 \section{Técnicas de Elicitação}
 Existem diversas técnicas que auxiliam no levantamento de requisitos, e todas elas partilham de um mesmo princípio: reduzir a grandeza de diversas dificuldades. Entre essas dificuldades estão:
\begin{description}
\item[$\bullet$] Cliente não ter uma idéia muito bem definida e construída do sistema que ele quer;
\item[$\bullet$] Cliente pode ter dificuldades em descrever o problema;
\item[$\bullet$] Cliente e analista podem ter diferentes pontos de vista do problema.
\end{description}
\subsection{Entrevista}
Entrevista é uma técnica tradicional que muitas vezes é utilizada como a técnica que dá o início do levantamento de requisitos. Os analistas discutem os problemas com diferentes stakeholders a fim de obter um entendimento das necessidades. É onde o analista dá bastante liberdade para o cliente expressar suas necessidades.Um dos pontos fortes dessa técnica é a comunicação direta do analista com o cliente, e um ponto fraco é a possível diferença cultural entre o entrevistador e o entrevistado. Existem algumas “dicas” que ajudam na realização dessa técnica:
\begin{description}
\item[$\bullet$] Informar a todos os envolvidos o tema a ser abordado;
\item[$\bullet$] Engenheiros de requisitos devem estar sujeito a sempre ouvir as ideias do cliente;
\item[$\bullet$] Informar a duração da entrevista (nunca mais que duas horas);
\item[$\bullet$] Explicar para o cliente como a entrevista será conduzida.
\end{description}
\subsection{Brainstorm}
Brainstorm é um termo do inglês que significa “tempestade de ideias”. É uma técnica que consiste em várias reuniões com intuito de geração de ideias. Todos os participantes são encorajados a expor qualquer tipo de ideia que eles tiverem, até mesmo as mais absurdas, pois uma das regras básicas dessa técnicas é não haver nenhum tipo de crítica a qualquer ideia exposta. Existem algumas “dicas” para essa técnica ser bem executada, são elas:
\begin{description}
\item[$\bullet$] Seleção dos participantes;
\item[$\bullet$] Explicar a técnica e as regras da sessão;
\item[$\bullet$] Sessão de no máximo 15 minutos;
\item[$\bullet$] Produzir uma boa quantidade de idéias.
\end{description}
Vale ressaltar que qualquer ideia que pareça absurda, é encorajada, pois são a partir delas que normalmente a criatividade vem.
\subsection{Justificativa}
Assim sendo, utilizaremos as técnicas Entrevista, Brainstorm e Imersão. De acordo com nossos estudos e leituras, definimos a técnica de Entrevista pelo motivo de geralmente ela ser a primeira técnica a ser utilizada para entender as necessidades do usuário e também por ser simples e direta. Brainstorm foi incorporado ao projeto visando uma melhor integração do grupo, pois o mesmo generaliza a participação dos membros, fazendo com que nenhuma ideia será inibida/criticada.


