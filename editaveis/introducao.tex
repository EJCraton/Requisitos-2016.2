\chapter{Introdução}

  Nesse documento está todo o planejamento, discussão e análises que envolvem o gerenciamento dos requisitos de um produto de
  software, os requisitos serão constantemente sendo aprimorado ao longo da disciplina de Requisitos de Software na Universidade
  de Brasília $-$ UnB.

  Este relatório está dividido em algumas partes excenciais para a boa realização do mesmo, na qual temos a parte de processos de
  Engenharia de Requisitos (ER), criado com base no SAFe (Scaled Agile Framework) e apoiado em um modelo de maturidade chamado MPS.Br,
  ambos serão detalhados mais a frente, temos um contexto do cliente e da empresa que iremos trabalhar, além do planejamento, abordagem,
  elicitação, gereciamento e a escolha da ferramenta que iremos utilizar.

  O cliente contemplado nesse projeto é uma empresa junior de geologia da UnB, e o objetivo principal das diversas atividades que a
  equipe de engenharia de software irá desenvolver está explicado nesse relatório de modo a alcançar o objetivo final que é produzir
  um produto de software de qualidade que agregue valor ao cliente atendendo todas as suas necessidades, essas atividades estão de
  acordo com os 5 pilares da Engenharia de Requisitos que são: Elicitação, análise e negociação, documentaçào, verificação e validação
  e gerência de requisitos.
